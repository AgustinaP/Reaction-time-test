\documentclass[11pt,a4paper,twocolumn]{article}
\usepackage[utf8]{inputenc}
\usepackage[spanish]{babel}
\usepackage{amsmath}
\usepackage{amsfonts}
\usepackage{amssymb}
\usepackage{graphicx}
\usepackage[left=2cm,right=1cm,top=1cm,bottom=1cm]{geometry}
\author{Nadia Agustina Pizarro}
\title{\textbf{Reaction Time Test}}
\date {\small{Viernes, 15 de Junio de 2018}}       % a confirmar.



\begin{document}

\twocolumn[
\begin{@twocolumnfalse}

\maketitle

\begin{center}
\abstract{\textit{El siguiente artículo se explicará, brevemente, el trabajo curricular realizado por la alumna Nadia Agustina Pizarro; presentado en el año 2017 como trabajo final para la asignatura \emph{Técnicas Digitales I}.\\
En simples palabras el trabajo consiste en realizar un test de elección para medir el tiempo desde que surge un estímulo visual pseudo-aleatorio hasta que el usuario pulsa el botón correspondiente. Este proyecto fue realizado mediante descripción de hardware utilizando el lenguaje VHDL, y luego transferido a la placa de puertas programable BASYS, que se encuentra disponible para estudiantes en el laboratorio de técnicas digitales de la Universidad Tecnológica Nacional Facultad Regional Córdoba. }}
\end{center}

\end{@twocolumnfalse}
]
\section*{Introducción}
	Popularmente se dice que una persona tiene buenos reflejos o que tiene malos reflejos en función de que sus tiempos de reacción sean cortos o largos respectivamente. Este tiempo dependerá de nuestra capacidad de \emph{percepción, procesamiento y respuesta}; nuestra capacidad de respuesta se verá afectado por la complejidad del estímulo, trastornos\footnote{cegueras, sorderas, bradipsiquia, Alzheimer, TDAH, acinesia, bradicinesia, Parkinson, hemiparesia u otras parálisis, esclerosis múltiple, corea de Huntington, traumatismos craneoencefálicos, ictus, daño axonal difuso (DAD), contusiones, daltonismo, personas impulsivas, entre otras.}, la familiaridad, el estado del organismo, la modalidad sensorial estimulada, entre otros.\\
Además de los factores explicados también afectará a nuestra respuesta cuando el test
es:
\begin{itemize}
\item[*] Simple: hay una única respuesta a un único estímulo.
\item[*] Elección: hay distintas respuestas a distintos estímulos.
\item[*] Selección: hay distintos estímulos, pero sólo tenemos que responder a uno de ellos.
\end{itemize}
El tiempo de respuesta está presente en la mayoría de las actividades de nuestro día a día. Que podamos interactuar correctamente con nuestro entorno y reaccionar ante los imprevistos que nos rodean depende directamente de nuestro tiempo de respuesta. De este modo, evaluarlo y conocer su estado puede ser de gran ayuda en diferentes ámbitos de la vida.
\paragraph{Palabras clave:}
Tiempo de reacción, capacidad de respuesta, estímulo, eficiencia.

\section*{Objetivos y Motivación}
El objetivo del trabajo cuando lo hice era presentar un proyecto integrador utilizando lógica programable y desarrollarlo utilizando un lenguaje de descripción de Hardware, aplicando todos los conocimientos de la materia para la aprobación directa de la materia.



\begin{flushright}

\end{flushright}
\end{document}